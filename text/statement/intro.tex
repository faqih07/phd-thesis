\chapter{Introduction}

% concrete
Concrete is a composite material composed of inclusions (gravel and sand aggregates) embedded in a cement (or similar binder) matrix and is the most widely used building material.
Therefore it has been in various contexts subjected to extensive research.
In practical civil engineering, concrete is usually idealized as a homogeneous isotropic material.
However, certain applications require description of concrete on lower than structural scales and heterogeneity (e.g, presence of aggregates) has to be taken into account.

The basic behavior and structural response of concrete structures may be described analytically (for example a beam structure in the elastic range).
Introducing more and more enhancements and features of the models leads to analytical unsolvability and numerical methods, usually with the help of computers, have to be introduced.

% coupling
Numerical simulations are an indispensable part of the current engineering and science development.
For different engineering areas there are different numerical methods used.
In solid phase mechanics, the leading methods are the finite element method (FEM) and the discrete (distinct) element method (DEM).

Usually, the solution is performed by a computer program, which is focused on a narrower or wider class of problems (such as solid mechanics, fluid dynamics, heat analysis, DEM etc.).
If a combination of two classes of problems is required (coupling of mechanical and heat analysis for instance), it is often possible to find a code allowing such approach.
However, in some cases, there exists no program that can solve the desired combination of problems.
One possible approach to deal with such situation would be to write a new or extend an existing program implementing the requested features.
Another possible approach would be to use existing independently developed codes, each one focused on a specific class of problems, and \quotes{glue} them together.

There are countless software programs for both FEM and DEM.
In this thesis, coupling of FEM code \OOFEM\ and DEM code \YADE\ is described and illustrated.


\section{Objectives of the thesis}
\begin{enumerate}

\item
To investigate basic properties of particle models, namely the relation between micro- and macroscopic elastic properties of random dense packings in terms of analytical formulas and results of numerical simulations.
Preparation and properties of random dense packings should be investigated beforehand.

\item
To develop open source tools for combination of the discrete element method and the finite element method.
Several classes of combination approaches together with simple examples should be addressed.

\item
To develop a mesoscale discrete element model for concrete.
The model should take into account the effect of aggregates and the interfacial transition zone (ITZ) between aggregates and the matrix.
The model should be validated against experimental data from available literature.

\end{enumerate}
